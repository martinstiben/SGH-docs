\documentclass[sigconf,nonacm]{acmart}

% Preamble común (paquetes neutros para IEEE/ACM/APA)
\usepackage[spanish, es-tabla]{babel}
\usepackage{iftex}
\ifPDFTeX
  \usepackage[T1]{fontenc}
  \usepackage[utf8]{inputenc}
\else
  \usepackage{fontspec}
\fi

\usepackage{microtype}
\usepackage{csquotes}
\usepackage{graphicx}
\usepackage{subcaption}
\usepackage{booktabs}
\usepackage{siunitx}
\usepackage{enumitem}
\usepackage{xcolor}
\usepackage{hyperref}
\usepackage{cleveref}

% Toggle para minted/listings
\newif\ifminted
\mintedfalse % Cambie a \mintedtrue si minted está disponible y configurado
\ifminted
  \usepackage[newfloat]{minted}
  \setminted{fontsize=\small, breaklines=true, autogobble=true}
\else
  \usepackage{listings}
  \lstset{
    basicstyle=\ttfamily\small,
    frame=single,
    breaklines=true,
    numbers=left,
    numberstyle=\tiny,
    showstringspaces=false,
    tabsize=2,
    extendedchars=true,
    inputencoding=utf8,
    literate={á}{{\'a}}1 {é}{{\'e}}1 {í}{{\'i}}1 {ó}{{\'o}}1 {ú}{{\'u}}1
             {Á}{{\'A}}1 {É}{{\'E}}1 {Í}{{\'I}}1 {Ó}{{\'O}}1 {Ú}{{\'U}}1
             {ñ}{{\~n}}1 {Ñ}{{\~N}}1 {ü}{{\"u}}1
  }
\fi

% Traducción de nombres para cleveref
\crefname{figure}{figura}{figuras}
\Crefname{figure}{Figura}{Figuras}
\crefname{table}{tabla}{tablas}
\Crefname{table}{Tabla}{Tablas}
% acmart ya configura natbib; activamos estilo autor-año (cambiar a numbers si lo prefiere)
\citestyle{acmauthoryear}

\AtBeginDocument{\providecommand\BibTeX{{\sc Bib}\TeX}}

\title[SGH]{SGH: Sistema de Gestión de Horarios}

\author{Juan Pablo Saavedra Chambo}
\affiliation{%
  \institution{Servicio Nacional de Aprendizaje (SENA), Regional Huila, Centro de la Industria, la Empresa y los Servicios (CIES), Ficha 2899747}
  \city{Neiva}
  \country{Colombia}}
\email{jpsaavedra32@soy.sena.edu.co}

\author{Jesus Ariel González Bonilla}
\affiliation{%
  \institution{Servicio Nacional de Aprendizaje (SENA), Regional Huila}
  \city{Neiva}
  \country{Colombia}}

\begin{abstract}
Imagina un sistema que simplifica la pesadilla de organizar horarios en escuelas y universidades. Ese es SGH, nuestro proyecto de gestión de horarios académicos, construido con tecnologías de vanguardia para asignar cursos, profesores y aulas de manera eficiente. Incluye una app web moderna hecha con Next.js y Tailwind CSS, que se ve genial y funciona en cualquier dispositivo; un backend sólido en Java usando Spring Boot, organizado en microservicios para que sea fácil de mantener y escalar; y una app móvil en React Native para gestionar todo desde el teléfono. Todo se conecta a través de APIs REST, asegurando que los datos fluyan sin problemas. SGH enfrenta retos como conflictos de horarios y optimización de recursos, usando herramientas open source accesibles. Más allá de eso, fomenta la integración con otros sistemas educativos, ayudando a crear una educación digital más inteligente y flexible.

Palabras clave: Gestión de horarios académicos; Next.js; Tailwind CSS; Arquitectura de microservicios; APIs REST; React Native; Spring Boot; Aplicación web; Optimización educativa.
\end{abstract}

\keywords{Gestión de horarios, Metodologías ágiles, Historias de usuario, Arquitectura modular, Spring Boot, Next.js, React Native, Aplicación web, Aplicación móvil}

\begin{document}
\maketitle

\section{Introducción}
Gestionar los horarios en un colegio o universidad es uno de esos desafíos que, si se resuelven bien, hacen que todo funcione mejor. Se trata de organizar las clases de manera inteligente: que no se solapen, que las aulas estén bien aprovechadas y que los profesores y estudiantes puedan concentrarse en lo importante, sin perder tiempo con planillas interminables. Los sistemas automatizados son clave aquí. No solo resuelven conflictos de horarios casi al instante, sino que permiten consultas rápidas y actualizaciones en tiempo real desde dispositivos Android o la web. Hoy en día, contar con una herramienta accesible desde el celular Android o la web no es un lujo, es una necesidad. Pero crear un sistema así no es sencillo. Hay que tener en cuenta muchos detalles: la disponibilidad de cada profesor, las preferencias de horario, además, que sea fácil de usar y seguro. Por eso SGH es una solución desarrollada para enfrentar estos retos de frente. Un sistema pensado para simplificar la gestión de horarios, usando tecnologías modernas que garantizan flexibilidad, seguridad y una experiencia intuitiva para todos.
\section{Marco teórico}
\subsection{Sistemas de Gestión de Horarios Académicos}
Imagina una herramienta clave en las instituciones educativas: un sistema de gestión de horarios académicos, creado para optimizar la asignación de recursos escasos como profesores, aulas y tiempos de clase. Estos sistemas unen datos diversos, desde la disponibilidad de docentes hasta restricciones curriculares, capacidades de salas y preferencias estudiantiles, usando algoritmos de optimización combinatoria para evitar conflictos y crear horarios funcionales.

En la literatura, destacan historias de éxito en universidades donde plataformas automatizadas han cortado drásticamente el tiempo de planificación manual, dejando a los administradores libres para centrarse en lo pedagógico \cite{desarrollo2025}. Por ejemplo, en campos logísticos parecidos, apps web y backend han mostrado su efectividad en optimizar recursos, adaptándose a imprevistos como cancelaciones o cambios de último minuto \cite{macias2021}.

La evolución de estos sistemas viene de la necesidad de lidiar con volúmenes de datos cada vez mayores y restricciones complejas. Investigaciones comparativas revelan que algoritmos heurísticos, como backtracking o programación lineal, son habituales para afrontar el problema de asignación de horarios, que es NP-completo en esencia \cite{torres2020}. Además, integrar datos en tiempo real, como actualizaciones de disponibilidad, demanda arquitecturas que manejen eventos asíncronos, impulsando el uso de patrones de mensajería y APIs eficaces.

\subsection{Arquitecturas de Software: Monolito vs Microservicios}
Cuando la escala de un sistema de gestión de horarios crece—ya sea por el número de usuarios, la complejidad de los datos o la frecuencia de actualizaciones—La arquitectura subyacente se vuelve crítica. Historicamente, muchos sistemas educativos comenzaron como monolitos, donde toda la lógica de aplicación reside en una sola unidad desplegable. Sin embargo, experiencias documentadas, como la migración de sistemas de transporte público a microservicios, demuestran que dividir la aplicación en servicios pequeños e independientes mejora la escalabilidad horizontal, reduce tiempos de respuesta y facilita el mantenimiento \cite{torres2020}. En un monolito, cambios en una parte del sistema pueden afectar al conjunto, mientras que en microservicios, cada servicio puede evolucionar de forma autónoma, permitiendo despliegues selectivos y recuperación de fallos granular.

En el contexto de SGH, se adopta un enfoque híbrido: un monolito modular en Spring Boot que simula microservicios lógicamente separados, permitiendo una futura transición sin refactorizaciones masivas. Esta decisión equilibra simplicidad en el desarrollo inicial con potencial de escalabilidad, alineándose con prácticas recomendadas para proyectos académicos de mediana escala \cite{torres2020}.

\subsection{Tecnologías Frontend: Next.js y Tailwind CSS}
El frontend web tiene un papel crucial en la usabilidad de sistemas como SGH, donde interfaces intuitivas son esenciales para administradores y profesores. Next.js, un framework basado en React, destaca por su capacidad de renderizado del lado del servidor (SSR) y generación de sitios estáticos (SSG), optimizando el rendimiento, mejorando el SEO y reduciendo tiempos de carga iniciales. Comparaciones entre frameworks JavaScript indican que Next.js ofrece una curva de aprendizaje amigable para equipos que conocen React, con una comunidad sólida que acelera el desarrollo \cite{lazuardy2022}. En proyectos parecidos, Next.js se ha usado para crear dashboards administrativos con navegación suave y componentes reutilizables, probando su idoneidad para apps de gestión educativa.

Complementando Next.js, Tailwind CSS proporciona un sistema de estilos utilitarios que permite diseños responsivos y modernos sin escribir CSS personalizado. Esta aproximación reduce el tiempo de desarrollo, mejora la consistencia visual y facilita la adaptación a dispositivos móviles. Experiencias en desarrollo web con Tailwind destacan su eficacia en la creación de interfaces limpias y accesibles, con soporte para temas oscuros y animaciones sutiles \cite{somi2021}. En el caso de SGH, la combinación de Next.js y Tailwind permite una interfaz web que responde rápidamente a interacciones del usuario, visualizando horarios en formatos como calendarios y tablas dinámicas.

\begin{figure}[h]
\centering
\includegraphics[width=0.8\columnwidth]{graphics/uno.png}
\caption{Modern Front End Web Architectures with React.Js and Next.Js.}
\label{fig:lazuardy}
\end{figure}

\begin{figure}[h]
\centering
\includegraphics[width=0.8\columnwidth]{graphics/dos.png}
\caption{User Interface Development of a Modern Web Application.}
\label{fig:somi}
\end{figure}

\subsection{Tecnologías Backend y Comunicación: Spring Boot y APIs REST}
En el backend, Spring Boot emerge como una opción madura para desarrollar servicios robustos y escalables. Su integración con JPA para persistencia de datos, Spring Security para autenticación, y soporte nativo para APIs REST lo hace ideal para sistemas como SGH, donde la lógica de negocio incluye algoritmos de optimización y manejo de entidades complejas. Spring Boot facilita la configuración de microservicios modulares, con controladores para endpoints REST, servicios para lógica de negocio y repositorios para acceso a datos, promoviendo una arquitectura limpia y mantenible.

La comunicación entre frontend y backend se basa en APIs REST, un protocolo estandarizado sobre HTTP que permite operaciones CRUD sobre recursos. Este enfoque desacopla clientes y servidores, facilitando evoluciones independientes y escalabilidad. En proyectos con React Native y Next.js, las APIs REST han probado su viabilidad para aplicaciones móviles y web, con baja sobrecarga y soporte para autenticación JWT y cifrado \cite{amodeo2013, macias2021}. En SGH, las APIs REST manejan solicitudes de creación, lectura, actualización y eliminación de cursos, profesores y horarios, asegurando interoperabilidad y facilidad de integración con futuros módulos.

\begin{figure}[h]
\centering
\includegraphics[width=0.8\columnwidth]{graphics/tres.png}
\caption{Principios de diseño de APIs REST.}
\label{fig:amodeo}
\end{figure}

\subsection{Tecnologías Móviles: React Native}
Para acceso multiplataforma, React Native permite desarrollar aplicaciones móviles nativas usando JavaScript, compartiendo lógica con el frontend web. Estudios comparativos entre frameworks móviles, como Flutter y React Native, destacan la superioridad de este último en términos de comunidad y facilidad de integración con ecosistemas React \cite{macias2021}. En SGH, React Native se utiliza para una app móvil que consume las mismas APIs REST, ofreciendo funcionalidades como visualización de horarios y notificaciones, optimizada inicialmente para Android con potencial de extensión a iOS.

\begin{figure}[h]
\centering
\includegraphics[width=0.8\columnwidth]{graphics/cuatro.png}
\caption{Estudio comparativo de los frameworks del desarrollo móvil.}
\label{fig:macias}
\end{figure}

\subsection{Seguridad y Privacidad en Sistemas Educativos}
La seguridad es un pilar fundamental en sistemas que manejan datos sensibles como información personal de estudiantes y profesores. Una autenticación coherente entre plataformas web y móvil, utilizando JWT para tokens seguros, es esencial; diseños unificados para React y React Native incluyen registro, login, recuperación de credenciales y manejo de tokens JWT \cite{ye2022}. Además, se recomienda control de acceso basado en roles, cifrado de comunicaciones vía HTTPS y cumplimiento de regulaciones como GDPR o leyes locales de protección de datos.

Más allá de lo técnico, principios de privacidad por diseño son cruciales en entornos educativos, donde el manejo ético de datos previene abusos. Aprendiendo de aplicaciones de rastreo de contactos, que incorporan enfoques de privacidad reforzada con consentimiento explícito y minimización de datos \cite{react2021}, SGH implementa medidas para asegurar que la gestión de horarios no comprometa la privacidad, como encriptación de datos en reposo y auditorías regulares.
\section{Metodología de investigación aplicada}
\subsection{Enfoque de desarrollo}
Para la gestión del proyecto se utilizó una metodología ágil inspirada en Scrum, con iteraciones semanales y priorización de funcionalidades, similar a prácticas en desarrollo con Java y Spring Boot \cite{arciniegas2025}. Los requerimientos funcionales incluyeron: registro de cursos, asignaturas, profesores y disponibilidades; generación automática de horarios; visualización y consulta de horarios; y exportación a PDF, Excel e imágenes. Estos requerimientos se especificaron mediante historias de usuario siguiendo el modelo INVEST (Independientes, Negociables, Valiosas, Estimables, Pequeñas y Verificables), permitiendo una definición iterativa y adaptable \cite{izaurralde2013}. Requerimientos no funcionales: seguridad en autenticación, usabilidad en interfaces y escalabilidad para múltiples usuarios.

La arquitectura seleccionada fue modular, con el backend en Spring Boot actuando como núcleo, el frontend web en Next.js para administración y el móvil en React Native para consultas. Esto permite separación de responsabilidades y facilidad de mantenimiento.

Desarrollo del backend (Spring Boot): Se implementó siguiendo el patrón MVC (Model-View-Controller), con controladores REST para CRUD de entidades (cursos, profesores, asignaturas, horarios), servicios de negocio y repositorios JPA. La generación automática de horarios utiliza algoritmos simples de asignación basados en disponibilidades. Se integró Spring Security con JWT para autenticación en el inicio de sesión. Servicios de exportación generan PDFs con iTextPDF, Excels con Apache POI y gráficos con JFreeChart.

Desarrollo de la aplicación web (Next.js): Construida con TypeScript, utiliza Axios para consumir la API REST. Incluye dashboards para gestión de entidades, generación de horarios y visualización con gráficos de Recharts. La interfaz es responsiva con Tailwind CSS y estilos personalizados en CSS.

Desarrollo de la aplicación móvil (React Native): Desarrollada con Expo para Android, permite login y consulta de horarios por curso o profesor. Utiliza navegación con React Navigation y almacenamiento local con AsyncStorage.

Pruebas e integración: Se realizaron pruebas unitarias en el backend con JUnit, pruebas manuales en interfaces y pruebas de integración end-to-end. Las APIs están documentadas con Swagger (OpenAPI) para facilitar el desarrollo y pruebas. La base de datos MySQL maneja persistencia, con Docker para orquestación en desarrollo.
\section{Implementación del software}
Describimos arquitectura, decisiones tecnológicas y \textit{pipelines}. Documentamos prácticas aplicadas: formateo, \textit{linting}, pruebas unitarias/integración, análisis estático (SAST), \textit{continuous delivery} y monitoreo. 

\subsection{Arquitectura del sistema}
La arquitectura implementada sigue las mejores prácticas de DevOps \cite{forsgren2018accelerate}, integrando automatización en todo el ciclo de desarrollo. % La \cref{fig:correlacion_devops} muestra la correlación observada entre el nivel de automatización y el rendimiento del equipo.

% \begin{figure}[htbp]
%     \centering
%     \includegraphics[width=0.46\textwidth]{graphics/correlacion_devops.pdf}
%     \caption{Correlación entre nivel de automatización y performance del equipo de desarrollo}
%     \label{fig:correlacion_devops}
% \end{figure}

\subsection{Comparación de tecnologías}
La selección de tecnologías se basó en criterios objetivos. La \cref{tab:frameworks} presenta una comparación detallada de los frameworks evaluados.

\input{tables/frameworks_comparison.tex}
\section{Evaluación y resultados}
Aplicación web funcional: Desarrollada en Next.js, permite gestión completa de entidades, generación de horarios y exportaciones. Incluye autenticación segura y dashboards intuitivos.

Backend robusto: En Spring Boot, maneja lógica de negocio, generación y exportaciones. Pruebas muestran rendimiento adecuado para instituciones medianas.

Aplicación móvil: En React Native para Android, facilita consultas rápidas de horarios, mejorando accesibilidad para estudiantes y profesores.

Generación automática: Algoritmos asignan horarios evitando conflictos, con historial de generaciones.

Exportaciones: Horarios exportables a PDF, Excel e imágenes, útiles para impresión y distribución.

Seguridad: JWT implementado, con roles de usuario (admin, coordinador).
\section{Discusión}
Los resultados permiten analizar fortalezas y áreas de mejora. La elección de Next.js y Tailwind CSS resultó óptima para interfaces modernas, con SSR mejorando SEO y rendimiento \cite{lazuardy2022, somi2021}. La arquitectura de microservicios modular facilita mantenibilidad, aunque la implementación monolítica inicial limita escalabilidad extrema \cite{torres2020}. APIs REST proporcionan interoperabilidad, pero podrían evolucionar a GraphQL para consultas más eficientes \cite{amodeo2013}. React Native añade flexibilidad móvil, alineándose con tendencias multiplataforma \cite{macias2021}.

Comparado con soluciones existentes, SGH se diferencia por su enfoque híbrido web/móvil y uso de Next.js/Tailwind, ofreciendo una alternativa accesible a sistemas comerciales. Limitaciones incluyen algoritmos de optimización básicos; futuras versiones podrían integrar una mejor automatizacion para mejores asignaciones. La privacidad se maneja adecuadamente, pero se recomienda auditorías externas para cumplimiento normativo.

En conclusión, SGH valida el stack tecnológico elegido, contribuyendo a la literatura sobre desarrollo educativo con tecnologías modernas.
\section{Conclusiones y trabajo futuro}
Este estudio demuestra que la aplicación sistemática de buenas prácticas de desarrollo produce mejoras medibles en calidad y eficiencia. Los resultados confirman las predicciones del marco DORA \cite{forsgren2018accelerate} sobre la correlación entre prácticas técnicas y rendimiento organizacional.

Las limitaciones incluyen el contexto específico del estudio y la necesidad de replicación en diferentes organizaciones. El trabajo futuro explorará la adaptación de estas prácticas a diferentes dominios y la automatización de su medición.

Proponemos una lista priorizada de prácticas y condiciones de aplicabilidad. Liberamos artefactos y un \textit{runbook} de adopción para facilitar la reproducibilidad y transferencia a la industria.

\section{Referencias}
\begin{enumerate}
\item Amodeo, E. (2013). Principios de diseño de APIs REST. Leanpub.
\item Lazuardy, M. F. S., \& Anggraini, D. (2022). Modern Front End Web Architectures with React.Js and Next.Js. International Research Journal of Advanced Engineering and Science, 7(1), 132-141.
\item Macías Vera, E. V. (2021). Estudio comparativo de los frameworks del desarrollo móvil "Flutter" y "React Native". Repositorio Nacional CEDIA.
\item Martín, H. (s.f.). Memoria TFM Héctor Martín. [Documento interno].
\item REACT. (2021). Rastreo de contactos en tiempo real y monitoreo de riesgos mediante rastreo móvil con privacidad mejorada. IEEE Xplore.
\item Somi, M. (2021). User Interface Development of a Modern Web Application. [Tesis].
\item Torres-Berru, Y., et al. (2020). Migración de un monolito a una arquitectura basada en microservicios. Dominio de las Ciencias, 6(2), 763-781.
\item Ye, X. P. (2022). Diseño e implantación de un sistema de autenticación multiplataforma para React y React Native. Universidad Politécnica de Madrid.
\item Desarrollo de un sistema de seguimiento de aplicaciones móviles basado en la nube para funciones de distribución logística de salida. (2025). Journal of Logistics Technology, 15(3), 50-60.
\item Tecnologías Front-end y Back-end en Tendencia. (2017). Recuperado de https://repository.ustaae8f-4b01-a066-849fcb70e15f
\item Especificando una arquitectura de software. (2020). Recuperado de https://revistas.udistrital.ed3
\item Practicante en lenguaje de programación JAVA y nuevas tecnologias. (2025). Recuperado de https://repositorio.utp.edu.co/entities/publication/192d53ab-1ac8-4829-ab8f-2639b995d120
\item Análisis prospectivo de la industria de desarrollo de software en Colombia. (2020). Recuperado de https://revistas.poligran.edu.co/index.php/puntodevista/article/view/1415
\item Una revisión comparativa de la literatura acerca de metodologías tradicionales y modernas de desarrollo de software. (2019). Recuperado de https://revistas.pascualbravo.edu.c6
\item Estudio sobre metodologías de desarrollo y su impacto en la productividad. (2020). Recuperado de https://revistas.udistrital.edu.co/index.php/tia/article/view/13364
\end{enumerate}

Ejemplo de petición POST para crear un curso:
\begin{verbatim}
POST /courses
Content-Type: application/json
Authorization: Bearer <token>
{
  "courseName": "1A",
  "gradeDirectorId": 123
}
\end{verbatim}
Respuesta: 200 OK con cuerpo \{"status": "OK", "message": "Curso creado correctamente"\}.

\appendix
\section{Descripción de componentes del sistema SGH}
Aplicación web (Next.js): Interfaz para administración, generación y visualización de horarios.

Servidor backend (Spring Boot): API REST para lógica de negocio y persistencia.

Aplicación móvil (React Native): Consulta de horarios en dispositivos Android.

Base de datos (MySQL): Almacenamiento de datos de cursos, profesores, horarios.

\section{Referencias}
1. Tecnologías Front-end y Back-end en Tendencia. (2017). Recuperado de https://repository.ustaae8f-4b01-a066-849fcb70e15f

2. Especificando una arquitectura de software. (2020). Recuperado de https://revistas.udistrital.ed3

3. Practicante en lenguaje de programación JAVA y nuevas tecnologias. (2025). Recuperado de https://repositorio.utp.edu.co/entities/publication/192d53ab-1ac8-4829-ab8f-2639b995d120

4. Análisis prospectivo de la industria de desarrollo de software en Colombia. (2020). Recuperado de https://revistas.poligran.edu.co/index.php/puntodevista/article/view/1415

5. Una revisión comparativa de la literatura acerca de metodologías tradicionales y modernas de desarrollo de software. (2019). Recuperado de https://revistas.pascualbravo.edu.c6

6. Estudio sobre metodologías de desarrollo y su impacto en la productividad. (2020). Recuperado de https://revistas.udistrital.edu.co/index.php/tia/article/view/13364

\end{document}