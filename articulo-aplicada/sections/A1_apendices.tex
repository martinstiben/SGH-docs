\appendix
\section{Descripción de componentes del sistema SGH}
Aplicación web (Next.js): Interfaz para administración, generación y visualización de horarios.

Servidor backend (Spring Boot): API REST para lógica de negocio y persistencia.

Aplicación móvil (React Native): Consulta de horarios en dispositivos Android.

Base de datos (MySQL): Almacenamiento de datos de cursos, profesores, horarios.

\section{Referencias}
1. Tecnologías Front-end y Back-end en Tendencia. (2017). Recuperado de https://repository.ustaae8f-4b01-a066-849fcb70e15f

2. Especificando una arquitectura de software. (2020). Recuperado de https://revistas.udistrital.ed3

3. Practicante en lenguaje de programación JAVA y nuevas tecnologias. (2025). Recuperado de https://repositorio.utp.edu.co/entities/publication/192d53ab-1ac8-4829-ab8f-2639b995d120

4. Análisis prospectivo de la industria de desarrollo de software en Colombia. (2020). Recuperado de https://revistas.poligran.edu.co/index.php/puntodevista/article/view/1415

5. Una revisión comparativa de la literatura acerca de metodologías tradicionales y modernas de desarrollo de software. (2019). Recuperado de https://revistas.pascualbravo.edu.c6

6. Estudio sobre metodologías de desarrollo y su impacto en la productividad. (2020). Recuperado de https://revistas.udistrital.edu.co/index.php/tia/article/view/13364