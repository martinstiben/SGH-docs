\section{Discusión}
Los resultados nos permiten analizar fortalezas y áreas de mejora. La elección de Next.js y Tailwind CSS resultó óptima para interfaces modernas, con SSR mejorando SEO y rendimiento \cite{lazuardy2022, somi2021}. La arquitectura de microservicios modular facilita la mantenibilidad, aunque la implementación monolítica inicial limita la escalabilidad extrema \cite{torres2020}. Las APIs REST proporcionan interoperabilidad, pero podrían evolucionar a GraphQL para consultas más eficientes \cite{amodeo2013}. React Native añade flexibilidad móvil, alineándose con tendencias multiplataforma \cite{macias2021}.

Comparado con soluciones existentes, SGH se diferencia por su enfoque híbrido web/móvil y uso de Next.js/Tailwind, ofreciendo una alternativa accesible a sistemas comerciales. Limitaciones incluyen algoritmos de optimización básicos; futuras versiones podrían integrar mejor automatización para asignaciones superiores. La privacidad se maneja adecuadamente, pero recomendamos auditorías externas para cumplimiento normativo.

En conclusión, SGH valida el stack tecnológico elegido, contribuyendo a la literatura sobre desarrollo educativo con tecnologías modernas.