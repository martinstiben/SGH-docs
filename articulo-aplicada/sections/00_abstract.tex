SGH es un proyecto para gestionar horarios escolares de forma sencilla y eficiente. Está pensado para simplificar la organización de los tiempos en instituciones educativas, aprovechando tecnologías modernas que garantizan un funcionamiento ágil y confiable. Cuenta con una aplicación web desarrollada con Next.js, un backend robusto creado en Java con Spring Boot, y una app móvil hecha en React Native para Android, para que estudiantes y profesores puedan consultar sus horarios en cualquier momento y desde dispositivos Android. Además, su arquitectura está diseñada de manera modular, lo que permite que el sistema crezca con facilidad y sea más fácil de mantener y mejorar con el tiempo.

Palabras clave: Gestión de horarios; Metodologías ágiles; Historias de usuario; Arquitectura modular; Spring Boot; Next.js; React Native; Aplicación web; Aplicación móvil.