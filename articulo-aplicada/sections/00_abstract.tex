SGH es un proyecto integral de gestión de horarios académicos diseñado con tecnologías modernas para optimizar la asignación de cursos, profesores y aulas en entornos educativos. El sistema se compone de una aplicación web desarrollada en Next.js con Tailwind CSS, proporcionando una interfaz de usuario responsiva, intuitiva y moderna; un backend robusto en Java con Spring Boot, estructurado en una arquitectura de microservicios para garantizar escalabilidad y mantenibilidad; y una aplicación móvil en React Native para acceso multiplataforma. La integración entre componentes se realiza mediante APIs REST, siguiendo principios de interoperabilidad y eficiencia en la comunicación de datos. Este proyecto aborda los desafíos de la gestión académica compleja, como la resolución de conflictos de horarios y la optimización de recursos, utilizando herramientas accesibles y de código abierto. SGH no solo facilita la planificación académica, sino que también promueve la interoperabilidad con otros sistemas institucionales, contribuyendo a una educación digital más eficiente y adaptable.

Palabras clave: Gestión de horarios académicos; Next.js; Tailwind CSS; Arquitectura de microservicios; APIs REST; React Native; Spring Boot; Aplicación web; Optimización educativa.