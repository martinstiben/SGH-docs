\section{Introducción}
La gestión de horarios académicos en instituciones educativas constituye un problema logístico de alta complejidad, que involucra la asignación óptima de recursos limitados como profesores, aulas y horarios de clases, con el objetivo de minimizar conflictos y maximizar la eficiencia operativa. En un contexto donde las instituciones educativas enfrentan crecientes demandas de personalización y flexibilidad, sistemas automatizados de gestión de horarios se convierten en herramientas esenciales. Por ejemplo, estudios han demostrado que plataformas de gestión educativa pueden reducir los tiempos de planificación manual en hasta un 40\%, liberando recursos humanos para tareas pedagógicas \cite{desarrollo2025}. Además, la integración con tecnologías en la nube y APIs REST permite crear ecosistemas interoperables, donde los datos académicos fluyen seamlessly entre sistemas administrativos, de matrícula y de evaluación.

A pesar de estos avances, implementar un sistema propio de gestión de horarios no es una tarea trivial. Requiere capturar y procesar datos heterogéneos, como la disponibilidad de profesores, restricciones de aulas, preferencias de estudiantes y requisitos curriculares; aplicar algoritmos de optimización para generar horarios viables; y presentar los resultados en interfaces amigables que permitan visualización, edición y exportación. Todo esto debe lograrse manteniendo bajos tiempos de respuesta, garantizando la seguridad y privacidad de los datos, y asegurando escalabilidad para instituciones de diversos tamaños. Consideraciones adicionales incluyen la usabilidad para usuarios no técnicos, la accesibilidad para personas con discapacidades, y la capacidad de integración con sistemas legacy existentes.

Este artículo presenta SGH (Sistema de Gestión de Horarios), un sistema de gestión de horarios académicos que aborda estos desafíos mediante un stack tecnológico moderno y accesible. El proyecto se estructura en tres componentes principales: una aplicación web desarrollada en Next.js con Tailwind CSS, que ofrece una interfaz responsiva y moderna para la gestión y visualización de horarios; un backend en Java con Spring Boot, diseñado con una arquitectura de microservicios para facilitar la escalabilidad y el mantenimiento; y una aplicación móvil en React Native, que extiende la funcionalidad a dispositivos móviles para acceso en tiempo real. La comunicación entre componentes se basa en APIs REST, inspiradas en principios de diseño interoperable \cite{amodeo2013}, lo que permite una separación clara de responsabilidades y facilita futuras expansiones. SGH se posiciona como una solución académica que no solo resuelve problemas inmediatos de planificación, sino que también sienta las bases para una gestión educativa más inteligente y conectada \cite{torres2020}.