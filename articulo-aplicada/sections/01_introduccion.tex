    \section{Introducción}

    Imagina un sistema que simplifica la pesadilla de organizar horarios en escuelas y universidades. Ese es SGH, nuestro proyecto de gestión de horarios académicos, construido con tecnologías de vanguardia para asignar cursos, profesores y aulas de manera eficiente. Incluye una app web moderna hecha con Next.js y Tailwind CSS, que se ve genial y funciona en cualquier dispositivo; un backend sólido en Java usando Spring Boot, organizado en microservicios para que sea fácil de mantener y escalar; y una app móvil en React Native para gestionar todo desde el teléfono. Todo se conecta a través de APIs REST, asegurando que los datos fluyan sin problemas. SGH enfrenta retos como conflictos de horarios y optimización de recursos, usando herramientas open source accesibles. Más allá de eso, fomenta la integración con otros sistemas educativos, ayudando a crear una educación digital más inteligente y flexible.

    Palabras clave: Gestión de horarios académicos; Next.js; Tailwind CSS; Arquitectura de microservicios; APIs REST; React Native; Spring Boot; Aplicación web; Optimización educativa.

    Pero construir un sistema así desde cero tiene sus propios retos. Necesitas reunir montones de información que muchas veces no encaja fácilmente: cuándo puede dar clases cada profesor, qué salones cumplen los requisitos, cuáles materias son prerrequisito de otras, y a veces hasta las preferencias de horario de los estudiantes. Una vez que tienes todo eso, hace falta un algoritmo lo suficientemente inteligente como para armar horarios que realmente funcionen sin crear conflictos. Y después está el tema de presentar todo eso de forma que sea fácil de entender y modificar para quien lo necesite. Por si fuera poco, el sistema tiene que ser rápido, mantener seguros los datos de la gente, poder crecer junto con la institución, ser fácil de usar incluso si no sabes mucho de computadoras, cumplir con normas de accesibilidad, y trabajar bien con los otros sistemas que ya tiene la institución.

    Pensando en todo esto fue que nació SGH (Sistema de Gestión de Horarios). Lo que hicimos fue crear una plataforma completa que responde a estas necesidades usando tecnologías que ya están bien establecidas y funcionan. El sistema tiene tres partes que trabajan juntas: primero, una aplicación web hecha con Next.js y Tailwind CSS que se ve bien y es práctica de usar desde cualquier navegador; segundo, un backend programado en Java con Spring Boot usando una arquitectura de microservicios, lo cual hace que sea más sencillo darle mantenimiento y agregarle funciones nuevas más adelante; y tercero, una app para celular desarrollada en React Native para que puedas consultar y manejar los horarios donde sea que estés. Estas tres partes se comunican entre sí mediante APIs REST diseñadas siguiendo estándares reconocidos \cite{amodeo2013}, de manera que cada componente puede funcionar por su lado pero siguen trabajando coordinadamente. SGH va más allá de ser solo un programa para hacer horarios; queremos que sea la base sobre la cual las instituciones puedan construir una forma más moderna e integrada de gestionar sus procesos educativos \cite{torres2020}.